% !TEX root = ../main.tex
\section{CRAB: Code Review Automated Benchmark}

\subsection{Overview}

In this section, we describe the construction of a high-quality Java-only dataset of code-review
triplets. Each triplet comprises three elements: the code snapshot before the review, the natural
language review comment and the revised code after that change has been applied. Our work supports
two complementary benchmarks: the first is review comment generation, asks a model to predict the
review comment given the pre-review code; the second is code refinement, a strict subset of the
first, asks a model to generate the revised code that implements the suggestion.


We focused on Java projects and aimed for high quality over quantity. That means we only kept
examples that are self-contained, testable, and likely to be meaningful. As a starting point, we
used the \textit{GHS} dataset and tool introduced by Dabić \etal~\cite{Dabic:msr2021data}, which
supports project sampling for Mining Software Repositories (MSR) studies by providing rich metadata
for over 700,000 GitHub repositories. Their platform made it possible to filter for Java
repositories with relevant properties like number of commits, license, and recent activity.

This section explains how we constructed the dataset, how we defined the two benchmark tasks
(comment generation and code refinement), and how we filtered and validated the data to ensure it’s
useful for evaluating future models.


\subsection{Data Source}

\subsubsection{Repository Sampling}

We began by querying the \textit{GHS} dataset~\cite{Dabic:msr2021data}, a large-scale repository
mining platform that indexes metadata for over 700,000 GitHub projects. Using the web interface at
\url{https://seart-ghs.si.usi.ch}, we applied two simple filters:

\begin{itemize}
	\item The repository must have had at least 100 pull requests.
	\item The date of the last commit must fall within the year 2025.
\end{itemize}

These criteria ensured that we targeted active, collaborative projects likely to have meaningful
code review activity. We exported the matching repository list in CSV format and used it as input
for our processing pipeline.

We applied minimal filtering beyond that. A small number of repositories were manually excluded
during processing, mostly due to extreme build times or authentication issues.

\subsubsection{Pull Request Selection}

For each selected repository, we searched for pull requests that reflect a clean and minimal review
interaction. Specifically, we looked for PRs that:

\begin{itemize}
	\item Were successfully merged into the main branch.
	\item Contain either one review comment or two, where the second is a reply from the PR author.
\end{itemize}

The goal is to capture cases where a reviewer leaves a single comment and the author either does not
respond in writing, or replies directly within the thread — which we interpret as an acknowledgment
of the feedback. This keeps the setup simple and avoids ambiguous review discussions.

We made this design choice because we realized early on that it would be difficult to create clean
triplets from pull requests with multiple comments. While we can always locate where each comment
appears in the code, it's much harder to untangle which subsequent code changes correspond to which
comment in an automatic way. This makes it challenging to extract multiple, independent \subCode,
\revComment and \revCode triplets from a single pull request. By focusing on PRs with only one
comment (plus an optional reply), we ensure that each PR maps to a single, self-contained instance.
This keeps the data interpretable and makes the benchmark more reliable.

Of course, even in these single-comment PRs, not all follow-up commits are guaranteed to directly
address the reviewer’s feedback. We handle these exceptions through manual validation, as detailed
later in Section~\ref{sec:manual-selection}.

\subsection{Automated Extraction Pipeline}
\label{sec:pipeline}
Transforming raw pull requests into usable training examples requires more than just scraping
metadata from GitHub. It involves carefully extracting code changes, associating them with reviewer
comments, validating their build and testability, and preserving the relevant context along the way.
To achieve this at scale and with high fidelity, we developed a robust, modular, and fully automated
pipeline. This system processes each pull request independently, performing all necessary
extraction, validation, and archival steps in a reproducible and fault-tolerant manner.

The following sections describe the guiding principles behind the pipeline, its overall
architecture, the key stages involved, and the mechanisms used for error handling and parallel
execution.

\subsubsection{Purpose and Design Philosophy}

The goal of the automated extraction pipeline is to transform raw pull requests from open-source
repositories into structured, high-quality dataset entries suitable for evaluation. To do this
effectively, the system must perform a complex sequence of steps, ranging from low-level repository
operations to high-level semantic filtering, while maintaining reliability, traceability, and
consistency at scale.

This pipeline is designed to be fully automated. Given a set of candidate repositories, it processes
each pull request independently from end to end, requiring no human intervention during execution.
This level of automation is critical for large-scale dataset construction, but it comes with
challenges: repositories vary widely in structure, quality, and tooling, and small inconsistencies
can derail processing unless handled robustly.

To address this, the pipeline is built with three guiding principles:
\begin{enumerate}
	\item \textbf{Isolation and Reproducibility.} All build and test steps are executed inside
	      Docker containers, ensuring that the results are not affected by the host system and can
	      be reproduced reliably.

	\item \textbf{Resilience to Failure.} Each processing step is isolated and allowed to fail
	      independently. A failure in one phase, such as test execution or coverage extraction, does
	      not invalidate the rest of the data. Instead, failures are captured and logged with
	      structured metadata so that partial but useful entries can still be retained.

	\item \textbf{Transparency and Debuggability.} Every step records its outcome and, in the case
	      of failure, the exact reason. This enables downstream analysis of failure rates and causes,
	      and facilitates future improvements to the pipeline itself.
\end{enumerate}

The result is a system capable of ingesting a wide variety of pull requests, gracefully handling
both clean and noisy cases, and producing a structured dataset where each entry includes not only
the final output but also a trace of what succeeded or failed during processing.

\subsubsection{High-Level Architecture}

The pipeline is structured as a sequential series of processing stages, each responsible for a
well-defined operation on a single pull request. These stages are executed one after the other, and
at each step, relevant metadata is recorded and potential errors are caught and stored. This design
promotes clarity, modularity, and traceability throughout the entire process.

At a high level, the pipeline begins by cloning the repository and extracting the relevant code and
comment information from the pull request. It then checks out the code states at the beginning and
end of the pull request, archives these snapshots, and determines whether the comment pertains to
code. If so, the pipeline continues by identifying the build system, attempting to compile the
project, executing the test suite, and extracting coverage metrics. Regardless of success or failure
in any step, a dataset entry is created and tagged with its processing outcome.

Although each pull request is treated independently from a logical standpoint, in practice, the
underlying repository state is shared on disk. For reasons of disk space efficiency, we do not
duplicate the repository for each pull request. As a result, pull requests from the same repository
are processed sequentially to avoid race conditions and state conflicts. However, because
repositories themselves are fully independent, the pipeline is parallelized at the repository level:
different repositories can be processed concurrently without interference. This strategy balances
isolation with scalability, allowing the system to make efficient use of compute resources while
maintaining correctness.

Figure~\ref{fig:pipeline} presents a simplified overview of the pipeline’s control flow. The diagram
groups the processing steps into three main phases: \textit{Setup}, \textit{Build \& Validation
	Assessment}, and \textit{Finalization}. There is also a decision point between \textit{Setup} and
\textit{Build \& Validation Assessment} that short-circuits unnecessary work when the comment is not
related to code.

\begin{itemize}
	\item \textbf{Setup}: This phase includes cloning the repository, extracting the diffs and
	      modified files, capturing the review comment, and archiving the relevant commits to preserve
	      a reproducible view of the pull request before and after the review.

	\item \textbf{Eligibility Check}: Before performing more expensive operations, the pipeline
	      checks whether the comment targets a Java file. If it does not, the system skips the
	      build phase entirely and proceeds directly to finalization, as test and build analysis
	      are not relevant.

	\item \textbf{Build \& Test Assessment}: For code-related comments, the pipeline identifies the build system,
	      attempts to compile the code, runs the test suite, and tries to generate code coverage using
	      JaCoCo. These steps are executed inside Docker containers for reproducibility.

	\item \textbf{Finalization}: Regardless of success or failure in the previous phases, the
	      pipeline saves the dataset entry along with rich metadata describing the processing outcome,
	      including reasons for failure when applicable.
\end{itemize}

This structure ensures a clean separation of concerns while maintaining extensibility and fault
tolerance. New capabilities or heuristics can be integrated into any phase with minimal disruption
to the rest of the system. The modular layout also enables robust error handling, allowing the
pipeline to continue processing even when individual pull requests encounter issues. This makes the
system scalable and resilient when applied to large and diverse sets of repositories.

\begin{figure}[!htbp]
	\centering
	\begin{tikzpicture}[
			node distance=1cm and 1cm,
			every node/.style={font=\small},
			startstop/.style={circle, draw, minimum size=1cm},
			process/.style={rectangle, draw, rounded corners, minimum width=4cm},
			decision/.style={diamond, draw, minimum width = 4cm, aspect = 2, align=center},
			arrow/.style={->, thick}
		]

		\node (start) [startstop] {Start};
		\node (prep) [process, below of=start, yshift=-.1cm] {Setup};
		\node (iscode) [decision, below of=prep, yshift=-.6cm] {Is comment\\code-related?};
		\node (validation) [process, right of=iscode, xshift=7cm] {Build \& Validation Assessment};
		\coordinate (mid) at ($ (iscode)!0.5!(validation) $);
		\node (save) [process] at ($(mid |- validation.south)+(0,-1cm)$) {Finalization};
		\node (end) [startstop, below of = save] {End};

		% arrows - linear path
		\draw[arrow] (start) -- (prep);
		\draw[arrow] (prep) -- (iscode);
		\draw[arrow] (iscode) -- node [above] {yes} (validation) ;
		\draw[arrow] (iscode) |- node [left] {no}(save) ;
		\draw[arrow] (validation) |- (save);
		\draw[arrow] (save) -- (end);
	\end{tikzpicture}
	\caption{Automated pipeline for processing a single pull request.}
	\label{fig:pipeline}
\end{figure}

\subsubsection{Key Pipeline Stages}

This section provides a more detailed walkthrough of the core stages outlined in
Figure~\ref{fig:pipeline}. Each pull request passes through the same structured sequence of
operations, with intermediate state and outcomes logged to facilitate inspection, debugging, and
dataset filtering.

\paragraph{Setup.} The pipeline begins by cloning the target repository, unless it already exists
locally. To avoid unnecessary downloads and to support resuming interrupted runs, previously cloned
repositories are reused. Once the repository is available, the system retrieves both the base and
merged commits of the pull request. These two commits serve as reference points to extract the code
diffs before and after the review comment.

The system then attempts to extract the content of all files modified by the pull request. This is
done using the GitHub API when possible, and directly from disk as a fallback when the API is
incomplete or fails to deliver consistent results. Special care is taken to record both the pre- and
post-merge versions of each file, as well as to ensure coverage data can be attached to the correct
source snapshot.

Review comments are extracted and filtered to retain only those that reference a specific file and
line range. This is necessary because GitHub occasionally serves comments without reliable line
anchoring. Comments without a valid line mapping are discarded early
in the process to reduce ambiguity downstream.

Both the base and merged states of the repository are archived in compressed form. These serve
distinct purposes. The base state (representing the code before any changes introduced by the pull
request) is primarily stored to provide broader context to models that may benefit from
access to the entire project structure. While the dataset already includes the full content of the
specific files involved in the pull request, some models may require a more holistic view of the
repository to make informed predictions. The merged state, on the other hand, is preserved for
evaluation purposes. In particular, it is used to test model-generated submissions in the context of
the code refinement task. Details of this evaluation setup are discussed further in
Section~\ref{sec:refinement}.

\paragraph{Eligibility Check.} Once the setup phase is complete, the system performs a quick check
to determine whether the review comment is attached to a source code file. If the comment points to
a non-code file (e.g. documentation, configuration files, or assets) it is extremely unlikely that
the suggested change involves actual code behavior. In such cases, running the full build phase,
that includes build, test, and coverage analysis, would be unnecessary and wasteful. To conserve
computational resources and reduce processing time, the pipeline bypasses that phase entirely for
these entries and moves directly to finalization. This early filtering mechanism ensures that only
potentially meaningful code-related changes undergo the more expensive analysis.

\paragraph{Build \& Test Assessment.} For comments that do target Java files, the pipeline enters
the build and test assessment phase. It first detects whether the repository uses Maven or Gradle by
inspecting known build configuration files. This allows it to spin up the appropriate Docker
container, preconfigured with the necessary environment for building and testing Java code.

Once inside the container, the pipeline attempts to compile the codebase. If the build succeeds, it
proceeds to run the test suite. If tests are detected and executed successfully, the system then
attempts to extract code coverage information using JaCoCo. If the project already includes JaCoCo
in its configuration, the system leverages that setup directly. Otherwise, it attempts to inject the
required configuration into the build process. While this injection is not guaranteed to succeed in
all cases, it enables coverage reporting in many repositories that would otherwise lack it.

After coverage reports are generated, the system analyzes them to determine whether the file
affected by the review comment is included in the results. This is done by extracting the fully
qualified class name from the file and searching for it in the coverage data. Due to the presence of
multi-module repositories and inconsistent directory structures, this association is not always
perfect. As discussed in Section~\ref{sec:multi-project-repo}, users must sometimes manually verify
that the reported coverage corresponds to the correct subproject.

\paragraph{Finalization.} At the end of the pipeline, the system records the outcome of each
processing step, including whether the pull request built and tested successfully, whether the
commented file was covered by tests, and whether the comment was code-related. If any step failed,
the reason is logged in structured metadata using a dedicated exception hierarchy. This allows for
filtering and debugging later on.

Every processed pull request results in a dataset entry, even if a phase failed. This is a
deliberate design choice: entries that fail the build phase may still contain useful review comments
for tasks such as comment generation. By capturing the full trace of each pull request—including
what succeeded, what failed, and why—the pipeline produces a dataset that is not only rich in
content but also transparent and debuggable.

\subsubsection{Failure Handling and Debuggability}

One of the key challenges in building a dataset from real-world repositories is handling the wide
range of inconsistencies, misconfigurations, and unexpected edge cases that arise in practice.
Repositories may have incomplete histories, broken builds, missing dependencies, or improperly
formatted comments. Rather than treat these cases as exceptions to be discarded, the pipeline is
designed to process them as first-class outcomes and record their failure modes explicitly.

Each stage of the pipeline is wrapped in fine-grained error handling logic. When a step fails
(whether due to a Git error, an unresolvable build issue, or malformed metadata) the failure is
caught, and a dedicated custom exception class is raised. These exceptions are structured,
categorized, and logged in the metadata of the corresponding dataset entry. This approach allows
each pull request to contribute useful diagnostic information, even when it does not yield a fully
valid instance.

This design serves two purposes. First, it enables comprehensive statistics about failure rates and
patterns across repositories. For example, one can measure how many entries fail at the build step,
how many fail due to missing line mappings in comments, or how often coverage extraction fails due
to JaCoCo injection issues. These insights help guide improvements to the pipeline and provide a
realistic picture of what can be expected when operating at scale.

Second, by retaining partially processed entries and clearly tagging them with their failure
context, the pipeline supports debugging, validation, and experimentation without requiring
re-processing from scratch. Entries that fail the build phase may still contain valuable
content for the comment generation task.

This robust failure management strategy ensures that the pipeline remains resilient in the face of
the inconsistencies inherent to open-source codebases. It allows for high-throughput processing
without sacrificing visibility into what went wrong, and it provides a mechanism for iterative
refinement over time.

\subsubsection{Parallelism and Scalability}

To process a large number of repositories efficiently, the pipeline is designed with parallel
execution capabilities. However, parallelism is applied at the repository level rather than the pull
request level. This distinction is critical due to the way repositories are managed on disk.

During processing, the build, test, and coverage phases operate directly on a local clone of the
repository. Creating separate copies of the repository for each pull request would incur a
substantial disk overhead, especially for large projects. To avoid this, the pipeline reuses a
single local clone per repository. This means that only one pull request per repository can be
processed at a time, ensuring consistency and preventing conflicts caused by concurrent checkouts or
file modifications.

To scale across repositories, the pipeline uses a multi-process architecture where multiple worker
processes handle different repositories in parallel. Each worker runs its own isolated pipeline
instance, with no shared state, and processes one repository at a time. This approach makes full use
of available compute resources while preserving correctness and reproducibility.

Parallelism also helps avoid bottlenecks caused by unbalanced workloads. Some repositories contain
only a handful of pull requests and can be processed quickly, while others (especially those with
extensive histories and complex build steps) take significantly longer. By distributing work across
multiple repositories concurrently, the system avoids having to wait on the slowest task to proceed.

However, true scalability is limited by another important factor: GitHub’s API rate limits. These
constraints, and how the pipeline addresses them, are discussed in the next section.

\subsubsection{Dealing with API Limits and Caching}

A critical bottleneck in any large-scale GitHub data processing pipeline is the platform's API rate
limit. GitHub imposes a cap of 5000 requests per hour per authenticated user. At first glance, this
may appear generous, but in practice, it is quickly exhausted due to how the GitHub Python client
operates.

The library adopts a lazy-loading design: most objects returned from the API are incomplete by
default and only fetch their full data when specific attributes are accessed. While this is
efficient for bandwidth and memory usage, it results in many additional network calls during regular
use. A single pull request may require dozens requests to fetch metadata, commit histories,
comments and file contents.

Fortunately, the library provides automatic handling of rate limit exhaustion. When the limit is
reached, it enters a waiting state and resumes once requests become available again. This makes it
safe to run the pipeline with multiple threads or processes: each one will automatically pause and
continue as needed without crashing or violating GitHub’s usage policies.

Despite this robustness, long pauses can be impractical when running large-scale crawls. To mitigate
this, the pipeline includes two layers of caching:

\begin{itemize}
	\item \textbf{HTTP-level caching:} The GitHub responses themselves are cached locally using an
	      auxiliary Python library, allowing the pipeline to skip API calls for data that has already
	      been retrieved. This not only saves bandwidth and time but also avoids rate limits entirely
	      for cached objects.

	\item \textbf{Pipeline-level checkpointing:} The system tracks which pull requests have already
	      been processed. If the pipeline is interrupted or restarted, it can pick up where it left
	      off without reprocessing completed work. This enables stop-and-resume operation, which is
	      especially valuable for long runs or incremental dataset builds.
\end{itemize}

While the GitHub Python library lacks thorough documentation, its internal behavior, once inspected,
proves reliable and well-suited for automation at scale. On several occasions, the implementation
had to be understood by directly reading its source code. Nonetheless, once these behaviors are
accounted for, the client operates smoothly, and the pipeline is able to run unattended for extended
periods, managing its own retries and pacing.

These design decisions—automated rate limit handling, layered caching, and task
checkpointing—combine to make the pipeline not just scalable in theory, but practically usable for
large-scale, real-world dataset construction.

\subsection{Build, Test \& Coverage}


\subsubsection{Motivation and Purpose}

Ensuring that each pull request can be built, tested, and instrumented for coverage is a key step in
constructing a dataset that goes beyond static code analysis. This phase is not just a validation
step for technical completeness, it also lays the groundwork for more robust evaluation. The ability
to compile and execute the code opens up the possibility of behavior-based assessment, which is
critical when evaluating code refinement models.

In contrast to similarity-based metrics such as CodeBLEU, which penalize variations in syntax even
when semantics are preserved, test-based evaluation allows for multiple valid implementations to be
accepted as long as they do not introduce regressions. While passing tests do not guarantee the
correctness of a change, they do strongly suggest that the change is not invalid. Failing tests, on
the other hand, clearly indicate broken behavior. This framing enables more realistic and flexible
model evaluation, better aligned with real-world development practices.

\subsubsection{Test Detection}

The first step is to determine whether the repository contains any form of automated
testing. This is accomplished by scanning the build configuration files for known testing libraries
(e.g., JUnit, TestNG, Mockito) and build system keywords such as \path{testImplementation} or
\path{functionalTests}. In addition to static analysis of the build file, the system also
inspects conventional test directories, such as \path{src/test/java} or \path{test/}, which
frequently contain unit or integration tests. This dual approach increases the reliability of test
detection across a broad set of repository structures.

\subsubsection{Execution Environment}

To ensure reproducibility, consistency, and isolation from the host system, all compilation and test
operations are executed within Docker containers. Two container environments are maintained,
corresponding to the two supported Java build systems: Maven and Gradle. Each container includes the
necessary tools, dependencies, and environment settings to execute a typical Java project. This
ensures that the results are not affected by host-specific variations, such as Java versions, OS
configuration, or missing libraries.

The architecture is designed with extensibility in mind. Build system support is modularized so that
new systems (such as Ant or Bazel) can be integrated by adding new handler classes and corresponding
container definitions. This separation of concerns facilitates future expansion to accommodate more
diverse types of software projects. Moreover, because the interface between the build system logic
and the rest of the pipeline is language agnostic, it is hypothetically straightforward to extend
support beyond Java and handle projects written in entirely different programming languages,
provided suitable build and test tooling exists for them.

\subsubsection{Coverage Generation}

Once a repository has been successfully built and tested, the next objective is to collect code
coverage data. This is primarily done using JaCoCo, a widely adopted tool for Java coverage
reporting. If the project is already configured with JaCoCo, the system attempts to run its existing
configuration directly. If this fails, typically due to the absence of the plugin, the system attempts
to inject JaCoCo manually into the build file, modifying the configuration to include the required
setup for coverage generation.

Care is taken to preserve the integrity of the build system: the original configuration is backed up
and restored if the injection fails. When successful, the injected configuration allows for the
generation of XML coverage reports, which are then parsed to extract coverage percentages for
individual files.

\subsubsection{Handling Multi-Project Repositories}

\label{sec:multi-project-repo}
A major challenge arises when dealing with large repositories that are organized into multiple
subprojects. These may be loosely coupled, deeply nested, or follow custom directory conventions.
Because of this structural diversity, it is difficult to reliably map a given Java file to a
specific coverage report. To approximate this mapping, the system extracts the fully qualified class
name of each commented file and searches for it across all available coverage files. If found, the
file is marked as covered.

This approach, while pragmatic, is not foolproof. It does not guarantee that the coverage report
belongs to the same subproject as the file in question. Given the variability in repository layout,
automatic resolution of this ambiguity is not feasible. Users are therefore advised to manually
verify coverage associations in cases where accuracy is critical.

\subsubsection{Enabling Flexible Evaluation}

By ensuring that the code in each pull request is buildable and testable, the dataset allows for a
richer model evaluation framework. Unlike static metrics that enforce a rigid similarity standard,
test-based validation supports functional correctness. This means that a model can propose diverse
implementations in response to reviewer comments, as long as the resulting code passes all tests. In
effect, this introduces a behavior-first evaluation protocol that aligns more closely with how
developers themselves judge correctness: not by form, but by function.

In summary, this phase adds significant value to the dataset by grounding it in real, executable
code, and by enabling flexible, semantically meaningful evaluation. Despite the technical challenges
involved—such as inconsistent configurations, incomplete test setups, and structural complexity—this
approach provides a much more powerful basis for training and evaluating automated code refinement
systems.

\subsection{Manual Selection}
\label{sec:manual-selection}

While the construction of the dataset is driven by automation to ensure scale and consistency, there
remain several aspects that fundamentally depend on human judgment. One of the most important among
these is the annotation of intent and follow-through: determining whether a reviewer’s comment
suggests a change, and whether that change was subsequently implemented in the pull request.

\subsubsection{Comment Classification}
The first layer of this process involves evaluating the nature of the review comment itself. Not all
comments are meant to trigger code modifications. Many comments are purely informational, offering
optional improvements or clarifications that do not warrant actual changes. Some comments are not
directly related to the current pull request but are instead left as notes for future work. These
typically refer to changes that should be made after the branch has been merged with another one,
but which cannot be carried out within the scope of the current pull request due to technical
constraints. Others might highlight a section of code without proposing any specific action. In
order to ensure that the dataset captures meaningful reviewer-author interactions, it is necessary
to distinguish between comments that suggest actionable modifications and those that do not.

A particularly nuanced category of comments consists of those framed as questions. While at first
glance they may appear less directive than imperative statements, questions often carry implicit
suggestions. These can be roughly divided into two main types. The first are rhetorical questions,
which are generally straightforward to interpret. These tend to imply strong disapproval or an
obvious recommendation, masked in the form of a question. For example, a reviewer might write,
\textit{``Can you move this declaration down closer to where it's used?''} or \textit{``Isn't this
	redundant with the previous check?''} In most cases, the intent behind such comments is clear: they
point out something that should be removed, refactored, or reconsidered.

The second type of question-based comment is more difficult to classify. These are genuine
inquiries, often posed in good faith, reflecting uncertainty or opening a discussion. They may
express doubt about a design choice, inquire about alternative implementations, or question whether
a particular solution addresses the intended problem. For instance, a reviewer might ask,
\textit{“Would it make sense to use a stream here instead of a loop?”} or \textit{“What happens if
	this input is null?”} The challenge with these types of comments is that they can be interpreted in
multiple ways. Some reviewers may expect an immediate change, while others are simply requesting
clarification. In many cases, the boundary between suggestion and inquiry is blurry, and
classification depends on context: including the tone of the comment, the review history, and even
the norms of the repository or organization. As a result, the decision to treat such a comment as a
change suggestion often comes down to the judgment of the person performing the manual annotation.

\subsubsection{Assessing Implementation of the Change}
Once a comment is identified as suggesting a change, the next step is to verify whether the change
was actually implemented. This involves inspecting the commits made after the comment was posted and
analyzing the diffs introduced. However, this is not as straightforward as it may seem. In modern
development workflows, especially those influenced by continuous integration and continuous
deployment (CI/CD) practices, it is common for the pull request branch to be updated with changes
from another branch, such as \texttt{main} or \texttt{dev}, before it is merged. This is often done
to resolve merge conflicts, bring in recent bug fixes, or ensure compatibility with the latest
version of the codebase. While such merges are necessary from an engineering perspective, they
introduce significant noise from the point of view of dataset construction.

When another branch is merged into the pull request, it can bring in a large number of changes that
are unrelated to the comment or even to the pull request itself. As a result, the diffs that appear
after the comment may include modifications that are completely irrelevant to the interaction being
studied. This creates a challenge: how can we isolate the changes that were made in response to the
comment from the rest?


\subsubsection{Selective Diff Curation}

To address this issue, the system incorporates a manual selection step where the user is shown the
diffs produced after the comment and is asked to select which hunks (blocks of code changes) are
relevant to the comment. This allows for precise filtering of the changes and helps construct a
cleaner, more focused dataset where only the diffs that represent a response to the review comment
are preserved.

In some cases, the situation is further complicated by the proximity of changes. Even when
irrelevant changes originate from merges or unrelated commits, they may occur in the same file or
even in adjacent lines to the relevant ones. This results in hunks that contain both the actual
response to the comment and unrelated code. Because the standard diff format groups nearby changes
into a single hunk, it becomes impossible to separate them without manual intervention. To deal with
such situations, the system provides the ability to edit the hunks directly. Users can manually trim
the diff to retain only the portions that directly address the reviewer’s feedback, discarding the
unrelated ones. This kind of fine-grained control is essential for preserving the integrity and
specificity of the dataset.

By performing this dual-level manual selection — first identifying whether a comment suggests a
change, and then isolating the relevant diff hunks — the dataset maintains a high degree of
fidelity. It captures meaningful reviewer-author interactions and filters out unrelated noise
introduced by collaborative development practices. This ensures that the final data is not only
accurate but also truly representative of how developers respond to feedback during the code review
process.

\subsubsection{Manual Selection as a Design Choice}

The decision to include a manual selection step was not an afterthought, but a deliberate design
choice. Prior work has shown that fully automated dataset construction methods often introduce a
significant amount of noise—whether due to misclassified comments, ambiguous code changes, or
incidental modifications unrelated to reviewer feedback. Rather than aiming for scale at the expense
of quality, we chose to prioritize precision and relevance. By hand-picking the instances we deemed
meaningful and representative, we ensured that the dataset would reflect true reviewer-author
interactions, with clearly identifiable suggestions and corresponding responses. This curated
approach allows for more reliable evaluation of models in tasks that require nuanced understanding
of code review dynamics.

\subsection{Paraphrases generation}
\subsection{Dataset Schema \& Serialization}

The dataset is designed to be both expressive and flexible. Each entry represents a single pull
request and contains a rich set of data capturing the code, the reviewer comments, the surrounding
context, and the repository’s response.

\subsubsection{Schema Overview}

Each entry in the dataset encapsulates multiple layers of information. At the core is the
\texttt{metadata} object, which includes identifying fields such as repository name, pull request
number, and commit hashes. It also contains metadata about the PR’s outcome, such as whether the
project built successfully, whether the file under review was covered by tests, and whether the
comment was judged to suggest a change.

The entry also stores:
\begin{itemize}
	\item The full set of \texttt{comments}, including body text, line ranges, and paraphrases.
	\item The \texttt{files} affected by the pull request, each annotated with its content before
	      and after the change, test coverage metrics, and whether it is code-related.
	\item Two sets of \texttt{diffs}: those between the opening of the PR and the reviewer comment
	      (``before''), and those occurring after the comment (``after'').
\end{itemize}

Together, these components allow for rich modeling of code review interactions over time.
Specialized views of this data, used for comment generation or code refinement tasks, can be derived
through structured filtering.

\subsubsection{Design Rationale}

The schema is built around the principle of modularity. Rather than entangling comment logic, diff
parsing, and code coverage into a flat structure, each concept is separated into its own dedicated
field or object. This makes it easy to reason about and operate on specific parts of the dataset
independently, for example, isolating all comments that suggest a change, or extracting only the
diffs related to code files.

\subsubsection{Serialization Formats}

The dataset supports several serialization modes, depending on the intended downstream task. These
include:
\begin{itemize}
	\item \texttt{full}: All fields are retained, giving a complete view of each entry.
	\item \texttt{comment\_gen}: Only includes entries where the reviewer comment suggests a change,
	      and retains just the code state and diffs before the comment. Useful to give as input to
	      a model for the comment generation task.
	\item \texttt{code\_refinement}: Filters to entries where the change was both suggested,
	      implemented and covered by tests. Includes the comment and the ``before'' diff, but omits
	      unrelated fields. Useful to give as input to a model for the code refinement task.
	\item \texttt{webapp}: A lightweight format that includes only metadata and comments, intended
	      for use in the webapp, so that it doesn't take long to load in memory.
\end{itemize}

Each mode produces a JSON file that is readable, compact, and suitable for different types of
experiments or tools. In addition, the serialization process allows for filtering out entries that
do not meet the criteria for inclusion in the dataset, such as comments that do not suggest a change
or pull requests that failed somewhere along the pipeline described in Section \ref{sec:pipeline}.
These “faulty” entries are not part of the intended dataset and are retained only as a form of
internal logging. They serve two main purposes: enabling detailed analysis of failure patterns
(e.g., identifying how many entries failed due to missing tests or broken builds), and helping
future efforts to improve the pipeline itself. The dataset, as serialized for
evaluation, is meant to consist exclusively of clean, validated entries that support the task of
comment generation or refinement.

\subsubsection{Use Case-Driven Filtering}

This flexibility allows researchers and engineers to tailor the dataset to the task at hand. For
example, a model trained to generate review comments may only require the pre-comment diff and
associated files. A refinement model, on the other hand, benefits from both the comment and the
changes made after it, with accurate annotation of whether the diff actually addressed the
suggestion.

The ability to serialize the dataset in multiple views avoids the need to rerun the full pipeline
every time a new task is proposed. It also supports experimentation and benchmarking across
different problem formulations while keeping the core data consistent.

\subsubsection{Extensibility and Programmatic Access}

Beyond serialization, the dataset class includes convenience methods for loading, filtering, and
mapping entries by ID. Internally, the logic is implemented in a way that is agnostic to language or
repository structure. This makes it easy to adapt the system to future extensions, such as
supporting other programming languages or integrating with other evaluation pipelines.

Overall, the dataset schema and its serialization logic reflect a balance between structure,
flexibility, and practical usability, ensuring that the data can evolve alongside the research it
supports.

\subsection{Dataset Statistics}
\subsection{Summary \& Deliverables}
