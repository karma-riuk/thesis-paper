% !TEX root = ../main.tex
\section{Introduction}
Code review is a cornerstone of modern software engineering, serving to enhance
code quality, maintainability, and collaboration among developers. Despite its critical
role, the process is expensive and labor-intensive, which has led to the development
of automated solutions to support key aspects, such as generating reviewer-like
comments identifying potential issues and refining the code in response to those comments.
While promising steps have been made using deep learning models, the lack of robust
benchmarks has hindered their reliable evaluation.
% particularly for tasks like review comment generation and code refinement.

Existing datasets, in fact, are automatically built from real code reviews
performed in open-source projects: they automatically try to identify and
extract the code before the review (\subCode), the reviewers' comments
(\revComment), and the code that implements those comments (\revCode). However,
this process is not entirely reliable. For example, several instances extracted
in this way have revised code that does not implement the associated comments
(reviewers' suggestions may be discussed elsewhere and ultimately not accepted)
or contain comments that do not accurately describe the problem found (such as
comments that simply refer to previous comments, e.g., "same here" \textcolor{red}{<-- (minor) questi tipi di commenti in realtà normalmente vengono fatti fuori da euristiche durante il cleaning. Forse metterei un esempio diverso}).

Furthermore, the metrics available today only allow a direct syntactic comparison with targets,
considering all those predictions that use different tokens than the target as wrong. For instance,
a comment generated by a model that expresses the same concept as the target but using
different words will be considered wrong. Likewise, the generated revised code that implements
the suggestions contained in the comment in an alternative way to the target will be considered
incorrect even if semantically correct.
Such inconsistencies and limitations obscure the real capabilities of the models.

%Although efforts to clean these datasets exist, the presence of problematic instances remains substantial.
% Consequently, evaluating models using these datasets often produces misleading
% results and impedes meaningful comparisons between approaches.

This thesis will try to address these challenges by introducing a high-quality
benchmark designed to evaluate code review automation approaches.
Specifically, it focuses on two critical tasks: (1) comment generation, \ie
generating natural language review comments for a given piece of code that emulate
human reviewers' comments by identifying problems in the code and making suggestions
for improvement, and (2) code refinement, \ie generating a piece of code that
implements the reviewers' suggestions for a given piece of code.

The proposed benchmark will consist of meticulously curated triplets (\subCode) \textcolor{red}{<-- why this (\subCode) here?}
where \revComment will be validated to highlight specific issues in \subCode and
\revCode will be validated to comprehensively address these issues.

To account for natural language variability, reviewers' comments will be accompanied by
alternative paraphrases that maintain their main intent; while to consider as correct also
those alternative code implementations but semantically equivalent to the target, code tests will
be included. These enhancements aim to set a new standard for assessing code review
automation tools by providing consistent and contextually meaningful data.

Furthermore, another goal of this thesis is the development of a web-based platform for researchers
to evaluate their models against the benchmark. We focus on the Java programming language and aim to
assemble a dataset with a size aligned with existing code generation benchmarks, which range from a
few dozen to a few hundred rigorously validated instances. This work aims to establish a solid
empirical foundation to drive the advancement of automated code review.

\subsection{Thesis Structure}

This thesis is organized into five chapters, each contributing to the overarching goal of advancing
automated code review through benchmark design and evaluation tooling.

\begin{itemize}
	\item \textbf{Chapter 2 - State of the Art}: This chapter provides a comprehensive overview of
	      existing research on code review automation. It covers key tasks such as code change
	      analysis, comment generation, code refinement, and review quality assessment. It also
	      surveys prominent benchmarks used to evaluate deep learning (DL) models for code-related
	      tasks, identifying their limitations and motivating the need for a more targeted benchmark.

	\item \textbf{Chapter 3 - CRAB: Code Review Automated Benchmark}: This chapter introduces CRAB,
	      a new benchmark designed to evaluate DL-based tools for automated code review. It details
	      the methodology for dataset construction, including repository selection, automated and
	      manual filtering, test and build validation, and paraphrase generation for reviewer
	      comments. The dataset schema, serialization formats, and curated statistics are also
	      discussed.

	\item \textbf{Chapter 4 - A Web App to Assess DL-Based Code Review via CRAB}: This chapter
	      describes the design and implementation of a web application that allows researchers to
	      interact with and evaluate models using the CRAB dataset. It presents the functional and
	      non-functional requirements, user interface, backend architecture, and evaluation pipelines
	      for both comment generation and code refinement.


	\item \textbf{Chapter 5 - Conclusions and Future Work}: This chapter reflects on the benchmark’s
	      development, highlights dataset statistics and coverage, and discusses limitations. It also
	      proposes future directions, like supporting additional languages, handling multiple comments
	      per PR, and running benchmarking experiments on state-of-the-art models.
\end{itemize}
